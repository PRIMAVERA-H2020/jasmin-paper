%% Copernicus Publications Manuscript Preparation Template for LaTeX Submissions
%% ---------------------------------
%% This template should be used for copernicus.cls
%% The class file and some style files are bundled in the Copernicus Latex Package, which can be downloaded from the different journal webpages.
%% For further assistance please contact Copernicus Publications at: production@copernicus.org
%% https://publications.copernicus.org/for_authors/manuscript_preparation.html


%% Please use the following documentclass and journal abbreviations for discussion papers and final revised papers.

%% 2-column papers and discussion papers
\documentclass[gmd, manuscript]{copernicus}

%% \usepackage commands included in the copernicus.cls:
%\usepackage[german, english]{babel}
%\usepackage{tabularx}
%\usepackage{cancel}
%\usepackage{multirow}
%\usepackage{supertabular}
%\usepackage{algorithmic}
%\usepackage{algorithm}
%\usepackage{amsthm}
%\usepackage{float}
%\usepackage{subfig}
%\usepackage{rotating}


\begin{document}

\title{PRIMAVERA multi climate model analysis at the JASMIN Super Data Cluster}


% \Author[affil]{given_name}{surname}

\Author[1]{Jon}{Seddon}
\Author[2]{Ag}{Stephens}
\Author[1]{Matthew}{Mizielinksi}

\affil[1]{Met Office Hadley Centre, FitzRoy Road, Exeter, EX1 3PB, UK}
\affil[2]{Centre for Environmental Data Aanalysis, RAL Space, STFC Rutherford Appleton Laboratory, Harwell Oxford, Didcot, OX11 0QX, UK}

%% The [] brackets identify the author with the corresponding affiliation. 1, 2, 3, etc. should be inserted.



\runningtitle{TEXT}

\runningauthor{TEXT}

\correspondence{Jon Seddon (jon.seddon@metoffice.gov.uk)}



\received{}
\pubdiscuss{} %% only important for two-stage journals
\revised{}
\accepted{}
\published{}

%% These dates will be inserted by Copernicus Publications during the typesetting process.


\firstpage{1}

\maketitle



\begin{abstract}
The PRIMAVERA project aims to develop a new generation of advanced and well-evaluated high-resolution global climate models. PRIMAVERA's Stream 1 simulations consist of seven different climate models being run at a standard and a higher resolution with common initial conditions and forcings to form a multi-model ensemble. The ensemble members were run a set of high performance computers across Europe. To allow the data from all models to be analysed it was transferred to the JASMIN super-data-cluster. At JASMIN the data was catalogued and details made available to users in the Data Management Tool (DMT). Users from across the project were able to query the available data using the DMT and to then login to JASMIN to analyse the data. Here we describe how the PRIMAVERA project used JASMIN's facilities to enable users to analyse this multi-model data set. We show how a facility such as JASMIN can be useful for similar multi-institute projects.
\end{abstract}


\copyrightstatement{Crown Copyright, Met Office}


\introduction  %% \introduction[modified heading if necessary]

The PRIMAVERA project aims to develop a new generation of advanced and well-evaluated high-resolution global climate models. One of PRIMAVERA's core components are the Stream 1 and Stream 2 simulations. These simulations consist of seven different climate models (AWI-CM-1-1, CMCC-CM2, CNRM-CM6-1, EC-Earth3P, ECMWF-IFS, HadGEM3-GC31 and MPI-ESM1-2) being run at their standard resolution (typically  250~km in the atmosphere and 100~km in the ocean) and at a higher resolution (25~km atmosphere and 8-25~km ocean). All models are run with common initial conditions and forcings using the HighResMIP protocol \citep{Haarsma2016}. The simulations were run on high performance computers (HPCs) across Europe. The scientists analysing the model outputs are based at 20 different institutes across Europe with assistance from other global scientists. Because these are global simulations at a high resolution, the total volume of data is expected to be around 1.6~petabytes (PB).

In the planning for PRIMAVERA a Data Management Plan (DMP) was developed to allow this data to be stored, analysed and archived. The JASMIN super-data-cluster was selected for this purpose. The Data Management Tool (DMT) software was developed to implement the DMP. 

\section{JASMIN}

JASMIN is a super-data-cluster that was first installed in 2012 \cite{lawrence2013storing}. It is funded by the Natural Environment Research Council (NERC) and the UK Space Agency. JASMIN is operated by the Science and Technology Facilities Council (STFC). It is located at the STFC Rutherford Appleton Lab, Harwell, UK. JASMIN's Phase 4 update in September 2018 added 38.5~PB of storage co-located with the existing over 4000 cores of compute all of which is tied together with a low latency network. RAL has a fast connection to JANET, the UK's academic network, which in turn is connected to the G\'{E}ANT European network, allowing fast transfer of data from HPCs to JASMIN. JASMIN additionally has access to RAL's tape library for offline storage of data.

The compute at JASMIN is split into interactive data analysis servers, the LOTUS batch processing system and some additional private cloud servers. Part of the storage is dedicated for the Centre for Environmental Data Analysis (CEDA) archive. The storage for individual projects is split into small chunks called group workspaces (GWS). 

\section{PRIMAVERA}

PRIMAVERA consists of eleven work packages (WPs). The Stream 1 and 2 simulations described here are produced and managed by two of the work packages. Several of the other WPs analyse these simulations and compare them and also compare them against existing simulations. The remaining WPs are carrying out model development work or running small simulations that don't need to be shared with other WPs. These remaining WPs had a small volume of storage on one of the GWS reserved for them. 

The Stream 1 and 2 simulations follow the HighResMIP protocol and will be submitted to the 6th phase of the Coupled Model Intercomparison Project (CMIP6) \citep{Eyring2016}. The Stream 1 simulations consisted of a single ensemble from member at standard and high resolution from each climate model. As a result of the first analysis of the Stream 1 simulations it was decided that a component of Stream 2 should be additional ensemble members from some of Stream 1 models, but with a reduced data output to minimise the volume of data generated.

\section{Data Management Plan}

% does D9.1 on the external webpage need adding to zenodo so that it can be referenced?



\subsection{JASMIN Resources Allocated}

As a result of the DMP, PRIMAVERA was allocated 440 terabytes (TB) of storage split into 5 GWS. A server in the JASMIN internal cloud was allocated, which was given the domain name prima-dm.ceda.ac.uk and HTTPS access from outside of JASMIN was granted to it.

\section{Data Management Tool}





\section{Data Transfer}

% include a discussion of the data transfer methods used and the rates achieved.


%\subsubsection{HEADING}


\conclusions  %% \conclusions[modified heading if necessary]
Without access to JASMIN then the data analysis and sharing in the PRIMAVERA project would have been more difficult...

%% The following commands are for the statements about the availability of data sets and/or software code corresponding to the manuscript.
%% It is strongly recommended to make use of these sections in case data sets and/or software code have been part of your research the article is based on.

\codeavailability{The PRIMAVERA Data Management Tool's source code can be found in a publicly available GitHub repository distributed under a BSD 3-Clause license \citep{Seddon2019}.} 



% \dataavailability{TEXT} %% use this section when having only data sets available


% \codedataavailability{TEXT} %% use this section when having data sets and software code available


% \sampleavailability{TEXT} %% use this section when having geoscientific samples available


% \videosupplement{TEXT} %% use this section when having video supplements available


%\appendix
%\section{}    %% Appendix A
%
%\subsection{}     %% Appendix A1, A2, etc.


\noappendix       %% use this to mark the end of the appendix section

%% Regarding figures and tables in appendices, the following two options are possible depending on your general handling of figures and tables in the manuscript environment:

%% Option 1: If you sorted all figures and tables into the sections of the text, please also sort the appendix figures and appendix tables into the respective appendix sections.
%% They will be correctly named automatically.

%% Option 2: If you put all figures after the reference list, please insert appendix tables and figures after the normal tables and figures.
%% To rename them correctly to A1, A2, etc., please add the following commands in front of them:

\appendixfigures  %% needs to be added in front of appendix figures

\appendixtables   %% needs to be added in front of appendix tables

%% Please add \clearpage between each table and/or figure. Further guidelines on figures and tables can be found below.



\authorcontribution{MM and AS developed the PRIMAVERA data management plan. JS and AS then developed and implemented the DMT and managed the data upload and availability. JS led the writing of this paper with assistance and editing from the other authors.} %% this section is mandatory for the journals ACP and GMD. For all other journals it is strongly recommended to make use of this section

\competinginterests{The authors declare that they have no conflict of interest.} %% this section is mandatory even if you declare that no competing interests are present

% \disclaimer{TEXT} %% optional section

\begin{acknowledgements}
The PRIMAVERA project is funded by the European Union's Horizon 2020 programme, Grant Agreement no. 641727. This work used JASMIN, the UK's collaborative data analysis environment (http://jasmin.ac.uk). The authors would like to thank all of the STFC staff who design and maintain JASMIN, without whom this work would not have been possible.
\end{acknowledgements}




%% REFERENCES

%% The reference list is compiled as follows:

%\begin{thebibliography}{}
%
%\bibitem[AUTHOR(YEAR)]{LABEL1}
%REFERENCE 1
%
%\bibitem[AUTHOR(YEAR)]{LABEL2}
%REFERENCE 2
%
%\end{thebibliography}

%% Since the Copernicus LaTeX package includes the BibTeX style file copernicus.bst,
%% authors experienced with BibTeX only have to include the following two lines:
%%
\bibliographystyle{copernicus}
\bibliography{jasmin-paper.bib}
%%
%% URLs and DOIs can be entered in your BibTeX file as:
%%
%% URL = {http://www.xyz.org/~jones/idx_g.htm}
%% DOI = {10.5194/xyz}


%% LITERATURE CITATIONS
%%
%% command                        & example result
%% \citet{jones90}|               & Jones et al. (1990)
%% \citep{jones90}|               & (Jones et al., 1990)
%% \citep{jones90,jones93}|       & (Jones et al., 1990, 1993)
%% \citep[p.~32]{jones90}|        & (Jones et al., 1990, p.~32)
%% \citep[e.g.,][]{jones90}|      & (e.g., Jones et al., 1990)
%% \citep[e.g.,][p.~32]{jones90}| & (e.g., Jones et al., 1990, p.~32)
%% \citeauthor{jones90}|          & Jones et al.
%% \citeyear{jones90}|            & 1990



%% FIGURES

%% When figures and tables are placed at the end of the MS (article in one-column style), please add \clearpage
%% between bibliography and first table and/or figure as well as between each table and/or figure.


%% ONE-COLUMN FIGURES

%%f
%\begin{figure}[t]
%\includegraphics[width=8.3cm]{FILE NAME}
%\caption{TEXT}
%\end{figure}
%
%%% TWO-COLUMN FIGURES
%
%%f
%\begin{figure*}[t]
%\includegraphics[width=12cm]{FILE NAME}
%\caption{TEXT}
%\end{figure*}
%
%
%%% TABLES
%%%
%%% The different columns must be seperated with a & command and should
%%% end with \\ to identify the column brake.
%
%%% ONE-COLUMN TABLE
%
%%t
%\begin{table}[t]
%\caption{TEXT}
%\begin{tabular}{column = lcr}
%\tophline
%
%\middlehline
%
%\bottomhline
%\end{tabular}
%\belowtable{} % Table Footnotes
%\end{table}
%
%%% TWO-COLUMN TABLE
%
%%t
%\begin{table*}[t]
%\caption{TEXT}
%\begin{tabular}{column = lcr}
%\tophline
%
%\middlehline
%
%\bottomhline
%\end{tabular}
%\belowtable{} % Table Footnotes
%\end{table*}
%
%%% LANDSCAPE TABLE
%
%%t
%\begin{sidewaystable*}[t]
%\caption{TEXT}
%\begin{tabular}{column = lcr}
%\tophline
%
%\middlehline
%
%\bottomhline
%\end{tabular}
%\belowtable{} % Table Footnotes
%\end{sidewaystable*}
%
%
%%% MATHEMATICAL EXPRESSIONS
%
%%% All papers typeset by Copernicus Publications follow the math typesetting regulations
%%% given by the IUPAC Green Book (IUPAC: Quantities, Units and Symbols in Physical Chemistry,
%%% 2nd Edn., Blackwell Science, available at: http://old.iupac.org/publications/books/gbook/green_book_2ed.pdf, 1993).
%%%
%%% Physical quantities/variables are typeset in italic font (t for time, T for Temperature)
%%% Indices which are not defined are typeset in italic font (x, y, z, a, b, c)
%%% Items/objects which are defined are typeset in roman font (Car A, Car B)
%%% Descriptions/specifications which are defined by itself are typeset in roman font (abs, rel, ref, tot, net, ice)
%%% Abbreviations from 2 letters are typeset in roman font (RH, LAI)
%%% Vectors are identified in bold italic font using \vec{x}
%%% Matrices are identified in bold roman font
%%% Multiplication signs are typeset using the LaTeX commands \times (for vector products, grids, and exponential notations) or \cdot
%%% The character * should not be applied as mutliplication sign
%
%
%%% EQUATIONS
%
%%% Single-row equation
%
%\begin{equation}
%
%\end{equation}
%
%%% Multiline equation
%
%\begin{align}
%& 3 + 5 = 8\\
%& 3 + 5 = 8\\
%& 3 + 5 = 8
%\end{align}
%
%
%%% MATRICES
%
%\begin{matrix}
%x & y & z\\
%x & y & z\\
%x & y & z\\
%\end{matrix}
%
%
%%% ALGORITHM
%
%\begin{algorithm}
%\caption{...}
%\label{a1}
%\begin{algorithmic}
%...
%\end{algorithmic}
%\end{algorithm}
%
%
%%% CHEMICAL FORMULAS AND REACTIONS
%
%%% For formulas embedded in the text, please use \chem{}
%
%%% The reaction environment creates labels including the letter R, i.e. (R1), (R2), etc.
%
%\begin{reaction}
%%% \rightarrow should be used for normal (one-way) chemical reactions
%%% \rightleftharpoons should be used for equilibria
%%% \leftrightarrow should be used for resonance structures
%\end{reaction}
%
%
%%% PHYSICAL UNITS
%%%
%%% Please use \unit{} and apply the exponential notation


\end{document}
